\documentclass{beamer}
%\usepackage{polski}
\usepackage[utf8]{inputenc}
\usepackage[OT4]{fontenc}

\usepackage{beamerthemesplit}
\usetheme{Boadilla}

\title{Modelowanie aukcji groszowych za pomocą systemu Jadex}
\author{Michał Dettlaff, Bartłomiej Kossakowski}
\date{\today}

\begin{document}

\frame{\titlepage}

%\section[Outline]{}
%\frame{\tableofcontents}

%\section{Introduction}
%\subsection{Overview of the Beamer Class}

\frame {
  \frametitle{Jadex}
  Implementacja modelu BDI:
  \begin{itemize}
    \item Belief
    \item Desire / Goal
    \item Intention / Plan
  \end{itemize}
}

\frame {
  \frametitle{Aukcje groszowe}
  Przykłady:
  \begin{itemize}
    \item podbij.pl
    \item fruli.pl
    \item za10groszy.pl
    \item swoopo.com
    \item BidRivals.com
  \end{itemize}
}

\frame {
  \frametitle{Aukcje groszowe}
  Każda oferta (podbicie, bid):
  \begin{itemize}
    \item podnosi cenę przedmiotu o 1 grosz
    \item przedłuża czas aukcji o 20 sekund
    \item kosztuje kupującego 50 groszy
  \end{itemize}
  \textit{Wartości liczbowe na przykładzie podbij.pl}
}

\frame {
  \frametitle{Protokół aukcji}
  \begin{block}{Rejestracja na stronie aukcyjnej}
    \begin{tabular}{ll}
      Sender       & \textit{kupujący} \\
      Receiver     & \textit{strona aukcyjna} \\
      Performative & subscribe \\
      Content      & \textit{nazwa użytkownika}
    \end{tabular}
  \end{block}
}

\frame {
  \frametitle{Protokół aukcji}
  Odpowiedź na prośbę o rejestrację
  \begin{block}{Potwierdzenie rejestracji}
    \begin{tabular}{ll}
      Sender       & \textit{strona aukcyjna} \\
      Receiver     & \textit{kupujący} \\
      Performative & agree \\
      Content      & \textit{cena podbicia} \\
                   & \textit{ilość podbić w pakiecie}
    \end{tabular}
  \end{block}
  albo
  \begin{block}{Odrzucenie prośby o rejestrację}
    \begin{tabular}{ll}
      Sender       & \textit{strona aukcyjna} \\
      Receiver     & \textit{kupujący} \\
      Performative & disagree
    \end{tabular}
  \end{block}
}

\frame {
  \frametitle{Protokół aukcji}
  \begin{block}{Dokupienie pakietu podbić}
    \begin{tabular}{ll}
      Sender       & \textit{kupujący} \\
      Receiver     & \textit{strona aukcyjna} \\
      Performative & request \\
      Content      & \textit{ilość pakietów} \\
                   & \textit{nazwa użytkownika}
    \end{tabular}
  \end{block}
}

\frame {
  \frametitle{Protokół aukcji}
  \begin{block}{Informacja o stanie aukcji}
    \begin{tabular}{ll}
      Sender       & \textit{strona aukcyjna} \\
      Receiver     & \textit{kupujący1, kupujący2, ...} \\
      Performative & cfp \\
      Content      & \textit{id aukcji} \\
                   & \textit{id przedmiotu} \\
                   & \textit{aktualna cena} \\
                   & \textit{aktualny wygrywający} \\
                   & \textit{pozostały czas}
    \end{tabular}
  \end{block}
}

\frame {
  \frametitle{Protokół aukcji}
  \begin{block}{Podbicie}
    \begin{tabular}{ll}
      Sender       & \textit{kupujący} \\
      Receiver     & \textit{strona aukcyjna} \\
      Performative & request \\
      Content      & \textit{id aukcji} \\
                   & \textit{nazwa użytkownika}
    \end{tabular}
  \end{block}
}

\frame {
  Odpowiedź na podbicie
  \begin{block}{Potwierdzenie podbicia}
    \begin{tabular}{ll}
      Sender       & \textit{strona aukcyjna} \\
      Receiver     & \textit{kupujący} \\
      Performative & confirm \\
      Content      & \textit{id aukcji}
    \end{tabular}
  \end{block}
  albo
  \begin{block}{Odrzucenie podbicia}
    \begin{tabular}{ll}
      Sender       & \textit{strona aukcyjna} \\
      Receiver     & \textit{kupujący} \\
      Performative & disconfirm \\
      Content      & \textit{id aukcji}
    \end{tabular}
  \end{block}
}

\frame {
  \frametitle{Aukcja dolarowa}
  \begin{itemize}
    \item przedmiotem aukcji jest banknot jednodolarowy
    \item dolar zostanie sprzedany temu, kto zaoferuje najwięcej
    \item ten kto będzie drugi w kolejności, też musi zapłacić swoją stawkę i nie dostanie nic w zamian
  \end{itemize}
}

\frame {
  \frametitle{Aukcja dolarowa}
  \begin{itemize}[<+->]
    \item paradoks teorii racjonalnego wyboru
    \item zawsze "racjonalną" decyzją jest podbicie oferty
    \item występuje irracjonalna eskalacja kosztów
  \end{itemize}
}

\frame {
  \frametitle{Aukcje groszowe}
  "A strange game. The only winning move is not to play."\newline
  \newline
  \textit{ \textemdash "Gry wojenne", film USA, 1983}
}
\end{document}
