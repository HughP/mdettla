% LaTeX Curriculum Vitae Template
%
% Copyright (C) 2004-2009 Jason Blevins <jrblevin@sdf.lonestar.org>
% http://jblevins.org/projects/cv-template/
%
% You may use use this document as a template to create your own CV
% and you may redistribute the source code freely. No attribution is
% required in any resulting documents. I do ask that you please leave
% this notice and the above URL in the source code if you choose to
% redistribute this file.
\documentclass[letterpaper]{article}
\usepackage{polski}
\usepackage[utf8]{inputenc}

\usepackage{hyperref}
\usepackage{geometry}

% Comment the following lines to use the default Computer Modern font
% instead of the Palatino font provided by the mathpazo package.
% Remove the 'osf' bit if you don't like the old style figures.
\usepackage[T1]{fontenc}
\usepackage[sc,osf]{mathpazo}

% Set your name here
\def\name{Michał Dettlaff}

% Replace this with a link to your CV if you like, or set it empty
% (as in \def\footerlink{}) to remove the link in the footer:
\def\footerlink{}

% The following metadata will show up in the PDF properties
\hypersetup{
  colorlinks = true,
  urlcolor = black,
  pdfauthor = {\name},
  pdfkeywords = {cv, informatyka},
  pdftitle = {\name: Curriculum Vitae},
  pdfsubject = {Curriculum Vitae},
  pdfpagemode = UseNone
}

\geometry{
  body={6.5in, 8.5in},
  left=1.0in,
  top=1.25in
}

% Customize page headers
\pagestyle{myheadings}
\markright{\name}
\thispagestyle{empty}

% Custom section fonts
\usepackage{sectsty}
\sectionfont{\rmfamily\mdseries\Large}
\subsectionfont{\rmfamily\mdseries\itshape\large}

% Other possible font commands include:
% \ttfamily for teletype,
% \sffamily for sans serif,
% \bfseries for bold,
% \scshape for small caps,
% \normalsize, \large, \Large, \LARGE sizes.

% Don't indent paragraphs.
\setlength\parindent{0em}

% Make lists without bullets
\renewenvironment{itemize}{
  \begin{list}{}{
    \setlength{\leftmargin}{1.5em}
  }
}{
  \end{list}
}

\begin{document}

% Place name at left
%{\huge \name}

% Alternatively, print name centered and bold:
%\centerline{\huge \bf \name}

\vspace{0.25in}

\begin{minipage}{0.65\linewidth}
  Michał Dettlaff \\
  ul. Kolorowa nr 7 \\
  84-105 Karwia \\
  516 187 321
\end{minipage}
\begin{minipage}{0.25\linewidth}
  Gdańsk, 19 maja 2009 \\
  \\
  \\
  \\
  \\
  \\
  \\
  \\
\end{minipage}

\paragraph{}
\noindent Szanowni Państwo,

\paragraph{}
\noindent W związku z ogłoszeniem o ofercie stażu zamieszczonym na stronie {\tt wakacyjnystaz.gdansk.gda.pl}, zgłaszam swoją kandydaturę.

\paragraph{}
\noindent Jestem studentem III roku informatyki na Uniwersytecie Gdańskim. Sądzę, że spełniam stawiane przez Państwa wymagania. Jestem dobrze obeznany ze standardem Java EE 5 i pokrewnymi technologiami. Potrafię tworzyć trójwarstwowe aplikacje przeznaczone na serwery spełniające ten standard, w czym solidne przygotowanie zapewniły mi przedmioty dotyczące Javy oraz Javy EE ukończone w trakcie studiów.

\paragraph{}
\noindent Umiejętność posługiwania się Środowiskiem Eclipse pozwala mi na sprawne tworzenie aplikacji. Od kilku lat używam niemal wyłącznie systemów linuksowych, co pozwoli mi na bardzo szybkie dostosowanie się do środowiska programowania stosowanego w Państwa firmie.

\paragraph{}
\noindent Moim celem jest praca na stanowisku, które pozwoli mi rozwinąć się jako programista poprzez efektywną pracę w zespole. Chętnie uczę się nowych technologii i stawiam sobie wymagające zadania. Oferuję ze swojej strony pracowitość, kreatywność i dbałość o jak najwyższą jakość efektów mojej pracy.

\paragraph{}
Mam nadzieję, że moja kandydatura wzbudzi Państwa zainteresowanie i będę miał szansę na zaprezentowanie się w dalszym etapie rekrutacji.

\paragraph{}
Z poważaniem,
\paragraph{}
Michał Dettlaff

\bigskip
\bigskip

% Footer
\begin{center}
  \begin{footnotesize}
    Wyrażam zgodę na przetwarzanie danych osobowych zawartych w ofercie pracy dla potrzeb realizacji procesu rekrutacji \\
    (zgodnie z ustawą z dnia 29.08.1997 r. o ochronie danych osobowych Dz. U. nr 133, poz. 883).
  \end{footnotesize}
\end{center}

\end{document}
