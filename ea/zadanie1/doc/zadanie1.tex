\documentclass[11pt]{article}
%\usepackage{polski}
\usepackage[utf8]{inputenc}
\usepackage[OT4]{fontenc}
\usepackage[pdftex]{graphicx}

\begin{document}
\title{Algorytmy ewolucyjne. Zadanie 1}
\author{Michał Dettlaff, Dariusz Kuziemski}
\maketitle

\section{Wprowadzenie}

\noindent
Zagadnienia polega na znalezieniu największego podzbioru wierzchołków
w grafie, którego żadne dwa wierzchołki nie są połączone ze sobą krawędzią 
(czyli wszystkie wierzchołki są niezależne od siebie nawzajem).

Celem jest znalezienie dobrego heurystycznego, tj. działającego w czasie 
wielomianowym algorytmu, który potrafi znaleźć rozwiązanie blisko optymalnego. 
W większości przypadków  heurystyki są zależne od problemu: heurystyka jest 
dobierana w zależności od własności problemu, który chcemy rozwiązać.

Dotychczas problemem zajmowali się między innymi Thomas Bäck i Sami Khuri 
w swojej pracy: „An Evolutionary Heuristic for the Maximum Independent Set Problem” 
(1994) zawartej w \textit{In Proceedings of the First IEEE Conference on Evolutionary Computation}.\newline
\begin{center}
\includegraphics[scale=0.5]{wiki9mis}
\end{center}

\newpage
\section{Metodologia}

\noindent
Problem rozwiążemy korzystając z Algorytmu genetycznego będącego naj-
popularniejszą reprezentacją klasy algorytmów losowego przeszukiwania 
bazujących na modelu ewolucji organicznej, tzw.  Algorytmów ewolucyjnych. 
Kanoniczne algorytmy genetyczne reprezentują indywidualne jednostki 
jako binarne wektory $ \vec{x} = (x_1,...,x_n) \in \{0,1\} $ stałej długości $n$.

Chromosom(potencjalne rozwiązanie będzie więc n-wymiarowym wekto-
rem $x$ ($n$ jest równe liczbie węzłów, $n=|V|$), przy czym jego i-ty element $x_i$
przyjmuje wartość 0 gdy $ i $-ty węzeł nie należy do $ V' $, $ x_i $ = 1 gdy $ v_i \in  V' $.

Niech $A$ = $[a_{ij}]_{n \times x}$ będzie macierzą incydencji grafu $G$, tzn.
\begin{center}
$a_{ij} = 
\left\{\begin{array}{rcl}
0 & gdy & (v_i,v_j) \notin E\\
1 & gdy & (v_i,v_j) \in E
\end{array} \right.
$
\end{center}

Funkcja dopasowania $ f: \{ 0,1 \} \to R $ która charakteryzuje 
rzeczywisty problem optymalizacji stanowi wskaźnik jakości dla 
jednostek. Wartości dopasowania są używane przez procedury 
selekcji ukierunkowując przeszukiwania w obszary z wyższym 
dopasowaniem mając nadzieję na otrzymanie lepszych wyników.
\begin{center}
$f(x) = \sum_{i=1}^n x_i - n \sum_{i>j} a_{ij} x_i x_j$
\end{center}
Przez $G=(V,E)$ oznaczamy graf gdzie $V$ to zbiór wierzchołków, a $E$ jest zbiorem 
krawędzi. Problemem jest wyznaczenie wyznaczenie zbioru $V' \subset V$ takiego, że
$\forall ~i,j \in V' $ krawędzi $\langle i,j \rangle \notin E $ maksymalny jest $|V|$. Jest to problem NP-trudny.

\newpage
\section{Wyniki działania algorytmu}

\noindent
Omówiony eksperyment został przygotowany przy użyciu algorytmu 
genetycznego z populacją o rozmiarze $ \mu/4$ = 9, z prawdopodobieństwem 
mutacji $p_m = 1/n$, krzyżowaniem 1-punktowym $p_c$ = 0.6, selekcję 
turniejową o rozmiarze turnieju 4, dla liczby iteracji 50, wartości 
zostały uśrednione po 30 powtórzeniach:\newline
%Domyślne parametry algorytmu:\newline
%Selekcja turniejowa o rozmiarze turnieju 4\newline
%Mutacja 1-punktowa z prawdopodobieństwem 1/n\newline
%Krzyżowanie 1-punktowe z prawdopodobieństwem 0.6\newline
%Rozmiar populacji $ \mu/4 = 112 $\newline
%\newline
%Najlepsze rozwiazanie po 30 powtórzeniach:\newline
%Wartość przystosowania = X\newline
%Średnie przystosowanie: X\newline
%Odchylenie standardowe: X\newline
%Czas wykonania: X s

\begin{tabular}{ | l | l |}
\hline
Iter. & Przystosowanie najlepszego w populacji \\ \hline
1 & -53,40 \\
2 & -34,80 \\
3 & -20,30 \\
4 & -13,40 \\
5 & -8,53 \\
6 & -4,83 \\
7 & -1,80 \\
8 & 0,27 \\
9 & 1,37 \\
10 & 3,60 \\
11 & 4,20 \\
12 & 3,93 \\
13 & 3,77 \\
14 & 4,53 \\
15 & 5,03 \\
16 & 6,17 \\
17 & 5,87 \\
18 & 6,07 \\
19 & 5,90 \\
20 & 6,33 \\
21 & 6,03 \\
22 & 5,30 \\
23 & 5,47 \\
24 & 6,00 \\
25 & 6,00 \\
26 & 3,77 \\
27 & 5,37 \\
28 & 5,47 \\
29 & 4,03 \\
30 & 4,67 \\

\hline
\end{tabular}

\begin{tabular}{ | l | l |}
\hline
Iter. & Przystosowanie najlepszego w populacji cd.\\ \hline
31 & 2,67 \\
32 & 3,07 \\
33 & 3,77 \\
34 & 5,23 \\
35 & 5,90 \\
36 & 6,73 \\
37 & 6,73 \\
38 & 5,30 \\
39 & 6,27 \\
40 & 4,53 \\
41 & 5,93 \\
42 & 6,00 \\
43 & 6,87 \\
44 & 5,37 \\
45 & 3,40 \\
46 & 5,43 \\
47 & 5,40 \\
48 & 5,73 \\
49 & 4,87 \\
50 & 5,90 \\
\hline
\end{tabular}

\includegraphics[scale=0.40]{1}\newline
Średnie przystosowanie: $-17,18$\newline
Odchylenie standardowe: $11,23$\newline
Czas wykonania: $1911 ms$\newline

Optymalne średnie przystosowanie zaczyna się stabilizować
już w okolicy 10-tego do 15-tego pokolenia i oscyluje 
eksloracje w tym stałym obszarze do końca działania algorytmu.
Odchylenie standardowe wynosi 11,23, czas działania 
programu 1911 ms.
 
%%%%%%%%%%%%%%%%%%%%%%%%%
\section{Pytania dodatkowe}

\begin{description}

\item[a)] Wyniki po zwiekszeniu rozmiaru populacji do $ \mu = \lfloor n/2 \rfloor $\newline

\includegraphics[scale=0.4]{2}\newline
%Wartość przystosowania = X\newline
%Średnie przystosowanie: X\newline
%Odchylenie standardowe: X\newline
%Czas wykonania: X s
Średnie przystosowanie: $-18,54$\newline
Odchylenie standardowe: $7,67$\newline
Czas wykonania: $2151 ms$\newline

Dzięki zwiększeniu rozmiaru populacji ewolucja najlepszego
rozwiązania szybciej osiąga obszar optymalnych rozwiązań 
oraz chętniej poszukuje rozwiązania w tym obszarze co obrazuje 
spadek współczynnika odchylenia standardowego, czas działania
programu wzrósł do 2151 ms.

%%%%%%%%%%%%%%%%%%%%%%%%%
\newpage
\item[b1)] Wyniki po zastosowaniu krzyżowania 2-punktowego\newline

\includegraphics[scale=0.4]{3}\newline
%Wartość przystosowania = X\newline
%Średnie przystosowanie: X\newline
%Odchylenie standardowe: X\newline
%Czas wykonania: X s
Średnie przystosowanie: $-20,31$\newline
Odchylenie standardowe: $11,49$\newline
Czas wykonania: $1424 ms$

Zastosowanie krzyżowania 2-punktowego wpływa znacząco na eksplorację 
obszaru rozwiązań - algorytm wykazuje tendencje do poszukiwania 
rozwiązań w szerszym obszarze, pokolenia są bardziej zróżnicowane.
Współczynnik odchylenia standardowego wzrósł (w niewielkim stopniu),
czas działania programu zmniejszył się do 1424 ms.

\newpage
%%%%%%%%%%%%%%%%%%%%%%%%%
\item[b2)] Wyniki po zrezygnowaniu z krzyżowania\newline

\includegraphics[scale=0.4]{4}\newline
%Wartość przystosowania = X\newline
%Średnie przystosowanie: X\newline
%Odchylenie standardowe: X\newline
%Czas wykonania: X s
Średnie przystosowanie: $-18,94$\newline
Odchylenie standardowe: $10,31$\newline
Czas wykonania: $1584 ms$

Po zrezygnowaniu z krzyżowania algorytm osiąga obszar
optymalnych rozwiązań wolniej, odchylenie standardowe 
również zmalało co wskazuje na zmniejszenie zdolności 
eksloracyjnych algorytmu. Czas działania programu 
zmniejszył się do 1584 ms.

%%%%%%%%%%%%%%%%%%%%%%%%%
\newpage
\item[c)] Wyniki po zastosowaniu selekcji rangowej\newline

\includegraphics[scale=0.4]{5}\newline
%Wartość przystosowania = X\newline
%Średnie przystosowanie: X\newline
%Odchylenie standardowe: X\newline
%Czas wykonania: X s
Średnie przystosowanie: $-42,87$\newline
Odchylenie standardowe: $21,56$\newline
Czas wykonania: $4892 ms$

Zastosowanie selekcji rangowej pozwala algorytmowi 
na bardziej skokową eksploracje, odchylenie standardowe
jest w tym przypadku najwyższe. Czas działania programu
wzrósł ponad dwukrotnie - 4892 ms.

%\newpage
\item[d)] Jak można usprawnić algorytm?\newline
Można inicjować populację poczatkową osobnikami nie całkowicie losowymi, ale
mającymi rozmiar zbioru niezależnego zbliżony do oczekiwanego w rozwiązaniu.
\item[e)] Porównać skuteczność algorytmu genetycznego z innym (znanym)
algorytmem wyznaczania maksymalnego zbioru niezależnego.\newline
Rozważmy prosty algorytm znajdowania maksymalnego zbioru niezależnego.
Zaczynamy od pojedynczego wierzchołka. Następnie znajdujemy wierzchołek,
który nie jest z nim bezpośrednio połączony i dodajemy go do naszego
zbioru. Następnie dodajemy wierzchołek, który nie jest bezpośrednio
połączony z żadnym z już dodanych i tak dalej, aż nie zostaną już
żadne wierzchołki do dodania.\newline
Algorytm ten ma złożoność wykładniczą i działa szybciej od algorytmu
genetycznego, ale podobnie jak algorytm genetyczny nie gwarantuje
znalezienia optymalnego rozwiązania.

\newpage
\item[f)] Praktyczne zastosowania rozwiązywanego problemu.\newline
Znajdowanie maksymalnych zbiorów niezależnych może mieć 
zastosowanie w wielu algorytmach związanych z problemami 
NP-trudnymi. \newline
Problem maksymalnego zbioru niezależnego znajduje 
zastosowanie w wielu dziedzinach, takich jak:
\begin{itemize}
\item Rozpoznawanie obrazów,\end{itemize}
\begin{itemize}
\item Biologia molekularna,\end{itemize}
\begin{itemize}
\item Zadanie harmonogramowania,\end{itemize}
\begin{itemize}
\item Oznaczanie map.\end{itemize}

\end{description}

\section{Bibliografia}

Thomas Bäck, Sami Khuri: An evolutionary heuristic for the maximum in-
dependent set problem. In: \textit{Proceedings of the First IEEE Conferen-
ce on Evolutionary Computation}, IEEE Press 1994, 531-535 
http://citeseerx.ist.psu.edu/viewdoc/summary?doi=10.1.1.54.624.8



\end{document}
