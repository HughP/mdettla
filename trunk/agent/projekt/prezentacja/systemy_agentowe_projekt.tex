\documentclass{beamer}

\usepackage{beamerthemesplit}

\title{Modelowanie aukcji groszowych za pomocą systemu Jadex}
\author{Michał Dettlaff, Bartłomiej Kossakowski}
\date{\today}

\begin{document}

\frame{\titlepage}

%\section[Outline]{}
%\frame{\tableofcontents}

%\section{Introduction}
%\subsection{Overview of the Beamer Class}

\frame {
  \frametitle{Jadex}
  \begin{itemize}
    \item Implementacja modelu BDI
    \item Belief
    \item Desire / Goal
    \item Intention / Plan
  \end{itemize}
}

\frame {
  \frametitle{Aukcje groszowe}
  \begin{itemize}
    \item podbij.pl
    \item fruli.pl
    \item za10groszy.pl
    \item swoopo.com
    \item BidRivals.com
  \end{itemize}
}

\frame {
  \frametitle{Aukcje groszowe}
  \begin{block}{Każda oferta (podbicie, bid)}
    \begin{itemize}
      \item podnosi cenę przedmiotu o 1 grosz
      \item przedłuża czas aukcji o 20 sekund
      \item kosztuje kupującego 50 groszy
    \end{itemize}
  \end{block}
  Wartości liczbowe na przykładzie podbij.pl
}

\frame {
  \frametitle{Protokół aukcji}
  \begin{block}{Rejestracja na stronie aukcyjnej}
    \begin{definition}
      \begin{tabular}{ll}
        Performative & subscribe \\
        Content      & nazwa\_agenta
      \end{tabular}
    \end{definition}
  \end{block}
  \begin{block}{Odpowiedź}
    \begin{definition}
      \begin{tabular}{ll}
        Performative & agree
      \end{tabular}
    \end{definition}
    albo
    \begin{definition}
      \begin{tabular}{ll}
        Performative & disagree
      \end{tabular}
    \end{definition}
  \end{block}
}
\end{document}
