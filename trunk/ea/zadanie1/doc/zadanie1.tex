\documentclass[11pt]{article}
\usepackage{polski}
\usepackage[utf8]{inputenc}
\begin{document}
\title{Algorytmy ewolucyjne. Zadanie 1}
\author{Michał Dettlaff}
\maketitle

\section{Wyniki działania algorytmu}

\noindent
Domyślne parametry algorytmu:\newline
Selekcja turniejowa o rozmiarze turnieju 4\newline
Mutacja 1-punktowa z prawdopodobieństwem 1/n\newline
Krzyżowanie 1-punktowe z prawdopodobieństwem 0.6\newline
Rozmiar populacji $ \mu/4 = 112 $\newline
\newline
Najlepsze rozwiązanie po 30 powtórzeniach:\newline
Wartość przystosowania = X\newline
Średnie przystosowanie: X\newline
Odchylenie standardowe: X\newline
Czas wykonania: X s

\section{Pytania dodatkowe}

\begin{description}

\item[a)] Wyniki po zwiększeniu rozmiaru populacji do $ \mu = \lfloor n/2 \rfloor $\newline
Wartość przystosowania = X\newline
Średnie przystosowanie: X\newline
Odchylenie standardowe: X\newline
Czas wykonania: X s
\item[b1)] Wyniki po zastosowaniu krzyżowania 2-punktowego\newline
Wartość przystosowania = X\newline
Średnie przystosowanie: X\newline
Odchylenie standardowe: X\newline
Czas wykonania: X s
\item[b2)] Wyniki po zrezygnowaniu z krzyżowania\newline
Wartość przystosowania = X\newline
Średnie przystosowanie: X\newline
Odchylenie standardowe: X\newline
Czas wykonania: X s
\item[c)] Wyniki po zastosowaniu selekcji rangowej\newline
Wartość przystosowania = X\newline
Średnie przystosowanie: X\newline
Odchylenie standardowe: X\newline
Czas wykonania: X s
\item[d)] Jak można usprawnic algorytm?\newline
Można inicjować populację początkową osobnikami nie całkowicie losowymi, ale
mającymi rozmiar zbioru niezależnego zbliżony do oczekiwanego w rozwiązaniu.
\item[e)] Porównać skuteczność algorytmu genetycznego z innym (znanym)
algorytmem wyznaczania maksymalnego zbioru niezależnego.\newline
Algorytm genetyczny działa wolniej.
\item[f)] Praktyczne zastosowania rozwiązywanego problemu.\newline
Znajdowanie maksymalnych zbiorów niezależnych może mieć zastosowanie w wielu
algorytmach związanych z problemami NP-trudnymi.\newline
Problem maksymalnego zbioru niezależnego znajduje zastosowanie w wielu dziedzinach, takich jak:\newline
Rozpoznawanie obrazów.\newline
Biologia molekularna.\newline
Zadanie harmonogramowania.\newline
Oznaczanie map.
\end{description}

\end{document}
