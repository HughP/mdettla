\documentclass[11pt]{article}
\usepackage{polski}
\usepackage[latin2]{inputenc}
\begin{document}
\title{Metody numeryczne. Zadanie B}
\author{Micha� Dettlaff}
\maketitle

\section{Tre�� zadania}

\noindent Napisa� program realizuj�cy metod� Newtona dla uk�adu r�wna�:
$$ \begin{array}[b]{rcl}
x + x^2 - 2yz & = & 0.1\\
y - y^2 + 3xz & = & -0.2\\
z + z^2 + 2x & = & 0.3
\end{array} $$
i przybli�enia startowego (0,0,0).

\section{Opis algorytmu}

Najpierw podstawiamy za $ \mathbf{x} $ wektor (0,0,0).\newline
Do wyznaczania kolejnych przybli�e� rozwi�zania korzystamy z wzoru:
\begin{displaymath} \mathbf{x_{k+1} = x_k + d}, \end{displaymath}
gdzie $ \mathbf{d} $ jest rozwi�zaniem r�wnania liniowego:
\begin{displaymath}
\mathbf{F'\left(x_k\right)d = -F\left(x_k\right)}
\end{displaymath}

\section{Najwa�niejsze struktury programu}

{\tt F(x)} - funkcja zwracaj�ca wektor warto�ci funkcji uk�adu r�wna� nieliniowych, obliczonych dla danego wektora $ \mathbf{x} $\newline
{\tt F\_d(x)} - funkcja zwracaj�ca macierz pochodnych cz�stkowych uk�adu funkcji F(x) dla danego wektora $ \mathbf{x} $\newline
{\tt x} - rozwi�zanie uk�adu (warto�ci x, y i z), pocz�tkowo r�wne przybli�eniu startowemu (0,0,0)

\section{Wyniki dzia�ania programu}

\noindent
$ \mathbf{x_n} \textrm{ - rozwi�zanie metod� Newtona} $

$$ \begin{array}[b]{lclll}
n&\vline& \mathbf{x_n}\\\hline
0&\vline& x=0 & y=0 & z=0\\
1&\vline& x=0.1 & y=-0.2 & z=0.1\\
2&\vline& x=0.024340317 & y=-0.1832997 & z=0.20872334\\
3&\vline& x=0.023363004 & y= -0.18167287 & z= 0.20941807\\
4&\vline& x=0.023363004 & y= -0.18167287 & z= 0.20941807
\end{array} $$

\end{document}
