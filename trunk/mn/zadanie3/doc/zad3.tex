\documentclass[11pt]{article}
\usepackage{polski}
\usepackage[latin2]{inputenc}
\begin{document}
\title{Metody numeryczne. Zadanie C}
\author{Micha� Dettlaff}
\maketitle

\section{Tre�� zadania}

\noindent Napisa� program testuj�cy metod� Eulera dla:
\begin{displaymath}
  y' = -y^2 - \frac{1}{4x^2+1},
\end{displaymath}
\begin{displaymath}
  y(0)=1; \textrm{ na przedziale [0,2].}
\end{displaymath}

\section{Opis algorytmu}

Do wyznaczania kolejnych przybli�e� rozwi�zania korzystamy z wzoru:
\begin{displaymath}
  y_{n+1} = y_n + h\cdot k
\end{displaymath}
gdzie
\begin{displaymath}
  h \textrm{ - krok algorytmu}
\end{displaymath}
\begin{displaymath}
  k = f(y_n)
\end{displaymath}
W naszym przypadku wz�r b�dzie mia� posta�:
\begin{displaymath}
  y_{n+1} = y_n + h\cdot \left( -y^2_n - \frac{1}{4x^2_n+1}\right)
\end{displaymath}

\section{Najwa�niejsze struktury programu}

{\tt F} - funkcja $ \displaystyle k = f(y_n) =
  -y^2_n - \frac{1}{4x^2_n+1} $\newline
{\tt y} - warto�� $ y_n $ liczona w kolejnych iteracjach\newline
{\tt h} - warto�� $ h $

\section{Wyniki dzia�ania programu}

\noindent
$ \{y\}^N_{i=0} \textrm{ - rozwi�zanie metod� Eulera} $

$$ \begin{array}[b]{lcl}
N&\vline&      y_N\\\hline
10&\vline&     0.082324\\
100&\vline&    0.220428\\
1000&\vline&   0.230309\\
10000&\vline&  0.231275\\
100000&\vline& 0.231372
\end{array} $$

\end{document}
